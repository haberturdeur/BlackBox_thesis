\documentclass{template/socthesis}

\usepackage[czech]{babel}
\usepackage[T1]{fontenc} % evropské uvozovky
\usepackage{csquotes}
\usepackage{xpatch}
\usepackage[author=,status=final]{fixme} % vkládání poznámek  
% dva módy (status): draft (poznámky se zobrazují v PDF) / final (poznámky se nezobrazují v PDF)

\usepackage{subcaption}
\usepackage[backend=biber,bibstyle=numeric,sorting=none,date=long,dateabbrev=false,texencoding=utf8,bibencoding=utf8,style=iso-numeric]{biblatex}
% \usepackage{amsmath}
\usepackage{enumitem}
\usepackage{xcolor}
\usepackage{listings}
\usepackage{float}
\usepackage{pdfpages}
\usepackage[toc,page]{appendix}


% \newfloat{code}{thph}{section}
% \floatname{code}{Ukázka kódu}

% \colorlet{shadecolor}{Gainsboro!50}
% \lstset{numbers=left, numberstyle=\tiny, stepnumber=1, numbersep=5pt}

\definecolor{codegreen}{rgb}{0,0.6,0}
\definecolor{codegray}{rgb}{0.5,0.5,0.5}
\definecolor{codepurple}{rgb}{0.58,0,0.82}
\definecolor{backcolour}{rgb}{0.95,0.95,0.92}

\DeclareQuoteAlias{german}{czech}
\MakeOuterQuote{"}

\lstdefinestyle{mystyle}{
    backgroundcolor=\color{backcolour},   
    commentstyle=\color{codegreen},
    keywordstyle=\color{magenta},
    numberstyle=\tiny\color{codegray},
    stringstyle=\color{codepurple},
    basicstyle=\ttfamily\footnotesize,
    breakatwhitespace=false,         
    breaklines=true,                 
    captionpos=b,                    
    keepspaces=true,                 
    numbers=left,                    
    numbersep=5pt,                  
    showspaces=false,                
    showstringspaces=false,
    showtabs=false,                  
    tabsize=4
}

\lstset{style=mystyle}
\def\lstlistingname{Ukázka kódu}
\def\lstlistlistingname{Seznam ukázek kódu}

\def\appendixpagename{Přílohy}
\def\appendixtocname{Přílohy}

\addbibresource{text.bib}

\titlecz{Software pro BlackBox}
\titleen{Software for BlackBox}
\author{Tomáš Rohlínek}
\field{18} % Obory SOČ: 1 - 18 (http://www.soc.cz/obory-soc/)
\school{Střední průmyslová škola a Vyšší odborná škola Brno, Sokolská, příspěvková organizace}
\mentor{Vojtěch Boček}
\mentorstatement{Vojtěchu Bočkovi}

% Změňte, pokud se liší
%\region{Jihomoravský}
% \placefooter{Brno 2017}

\begin{document}

\maketitle

\makecopyrightstatement{V~Drásově}

\makethanks{Děkuji svému školiteli Vojtěchu Bočkovi za obětavou pomoc, podnětné připomínky a~trpělivost, kterou mi během práce poskytoval.}


\pagestyle{empty}
\enlargethispage{20mm}

\section*{Anotace}
Cílem této práce je vytvoření SDK pro práci a~výuku na IoT stavebnici BlackBox postavené na platformě ESP32. Toto SDK se skládá ze tří částí: 

Hardwarové knihovny, které umožňují práci s jednotlivými periferiemi BlackBoxu 
a umožňují plně využít všech jeho možností. 

Výukové rozhraní slouží k podpoře výuky programování, speciálně na poli hobby robotiky. 
 
Herní rozhraní pro snadnou uživatelskou práci s BlackBoxem -- umožňuje jednoduše naprogramovat do BlackBoxu vlastní hry a nebo nahrát do něj hry už hotové. 

Také byla sepsána dokumentace k tomuto software v českém a anglickém jazyce. 



\subsection*{Klíčová slova}
BlackBox; IoT; API; SDK; C/C++; ESP32; zážitková akce; hry; mikrokontrolér

% \vspace{10mm}

\section*{Annotation}
The aim of this thesis is to create an SDK for development and education on IoT kit BlackBox built on the ESP32 platform. This SDK consists of three parts:

Hardware libraries which allow you to work with individual BlackBox peripherals,
as well as allow you to take full advantage of its capabilities.

The educational API is used to enhance programming teaching, especially in the field of hobby robotics.
 
Game API which gives user easy approach to BlackBox -- allows you to easily program your own games into the BlackBox or load pre-made games.

Documentation for this software was also written in Czech and English.

\subsection*{Keywords}
BlackBox; IoT; API; SDK; C/C++; ESP32; adventure; games; microcontroller

\newpage
\pagestyle{plain}

\tableofcontents % vysází obsah

%%% Začátek práce
\setcounter{figure}{0}
\setcounter{table}{0}
\newpage

%%% Úvod
\input{01-uvod.tex}

\input{10-hardware.tex}

\chapter{Technologie}

\section{Framework}

Pro vývoj na microcontroller ESP32 se používají hlavně dva frameworky a~to ESP-IDF \cite{ESP-IDF} a Arduino \cite{arduino}, oba jsou pro jazyk C/C++.

\subsection{ESP-IDF}

ESP-IDF, nebo také Espressif IoT Development Framework, je oficiální framework od výrobce ESP32, firmy Espressif Systems \cite{espressif}.
Je psaný pro vývoj v~jazyce C a~C++, samotný je psaný v~jazyce C.
Obsahuje několik úrovní abstrakce od přímé práce s~registry pro uživatelsky přívětivé API.

\begin{minipage}{\linewidth}
\lstinputlisting[language=C++, caption=ESP-IDF]{code/espidf.cpp}
\end{minipage}

\subsection{Arduino}

\enlargethispage{5mm}
Arduino je pro začátečníka jednoduší na pochopení a~na práci než ESP-IDF, protože funguje jako úroveň abstrakce nad ESP-IDF.
Jeho obrovskou předností a~zároveň jeho největší limitací je jeho kompatibilita pro množství naprosto rozličných platforem a~architektur sahající až po 8-bitové mikročipy ATtiny.~Bohužel stabilní vývojová větev Arduina pro ESP32 používá zastaralou verzi ESP-IDF, která má některá omezení, kupříkladu nepodporuje C++~17.

\begin{minipage}{\linewidth}
    \lstinputlisting[language=C++, caption=Arduino]{code/arduino.cpp}
\end{minipage}

\subsection{Další frameworky}

Samozřejmě existují i~další frameworky, kupříkladu MicroPython \cite{uPython} a~CircuitPython \cite{circuitPython}, které přivádí jazyk Python na mikrokotrolery, nebo Espruino \cite{espruino}, které dělá to samé pro JavaScript.

\subsection{Výběr}

Pro BlackBox jsem se rozhodl použít přímo ESP-IDF.
K~tomuto rozhodnutí mě vedl fakt, že moje knihovna bude sloužit jako úroveň abstrakce, tudíž Arduino mezivrstva je v~podstatě zbytečná, zároveň tím získám lepší kontrolu nad ESP32.
Protiargumentem by mohlo být množství knihoven dostupných pro Arduino.
Bohužel tyto knihovny nedodržují žádný společný rámec a~většina z~nich není uzpůsobena pro práci na více jádrových procesorech, jakým ESP32 je.\footnote{Nejsou thread safe.}
Zároveň toto umožní zpětnou kompatibilitu s oběma frameworky.

\section{Použité knihovny}

\begin{itemize}
    \item SmartLeds \cite{SmartLeds} --
        Knihovna pro interakci s chytrými led WS2812 pomocí hardwarové periferie RMT na ESP32. % fixme odkazy, zdroje
    \item Eventpp --
        Knihovna pro jednoduchou práci s událostmi.
\end{itemize}

\input{30-architektura.tex}

\input{31-vyukoveAPI.tex}

\input{32-herniAPI.tex}

\input{33-knihovny.tex}

\input{40-dokumentace.tex}

\input{100-zaver.tex}



% \listoftables
% \addcontentsline{toc}{section}{Seznam tabulek}

% \listoflistedequation
% \addcontentsline{toc}{section}{Seznam rovnic}



\begin{appendices}
    \newpage

    \printbibliography[title=Literatura]
    \addcontentsline{toc}{section}{Literatura}

    \listoffigures
    \addcontentsline{toc}{section}{Seznam obrázků}

    \lstlistoflistings
    \addcontentsline{toc}{section}{\lstlistlistingname}
    
    
    \begin{figure}
        ~ 
    \end{figure}
    \clearpage
    \phantomsection

    \includepdf[pagecommand=\thispagestyle{empty} \addcontentsline{toc}{section}{Zpráva o využití BlackBoxu}, page=1]{Socka-BlackBox-pouzitelnost.pdf}
    \includepdf[page=2]{Socka-BlackBox-pouzitelnost.pdf}
\end{appendices}

\end{document}
